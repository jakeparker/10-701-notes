\documentclass[12pt, a4paper, oneside, fleqn]{notes}
\usepackage[style=boxed]{preamble}
\usepackage{letters}

\usepackage{etex}

\usepackage[linktoc=all]{hyperref}
\hypersetup{pdftex,colorlinks=false}
\usepackage{hypcap}

\usepackage{enumerate, enumitem, extramarks, linegoal, graphicx, fancyhdr}
\usepackage{amsmath}

\newcommand{\Title}{Lecture Notes}
\newcommand{\Date}{2018}
\newcommand{\Class}{10--701}
\newcommand{\ClassName}{Introduction to Machine Learning}
\newcommand{\Name}{Jake Parker, ...}

\pagestyle{fancyplain}
\lhead{\fancyplain{}{\Class}}
\chead{\fancyplain{}{\ClassName}}
\rhead{\fancyplain{}{\Title}}

%%%%%%%%%%%%%%%%%%%%%%%%%%%%%%%%%%%%%%%%%%%%%%%%%%%%%%%%%%%%%%%%%%%%%%%%%%%%%%%%

\begin{document}

\tableofcontents

% [1] Foundations
\chapter{Foundations}
\rhead{\fancyplain{}{\Title \; 00}}
\section{}
 % {data, algorithms, tasks}
\include{lecture_notes_01} % {estimators, guarantees (learning theory), mle}
\rhead{\fancyplain{}{\Title \; 02}}
\section{}
 % {bayesian estimation, map}
\rhead{\fancyplain{}{\Title \; 03}}
\section{}
 % {model-free methods, decision theory, bayes risk}
\include{lecture_notes_10} % {generalization, model selection}
\rhead{\fancyplain{}{\Title \; 26}}
\section{}
 % {statistical guarantees for empirical risk minimization}

% [2] Regression
\chapter{Regression}
\rhead{\fancyplain{}{\Title \; 04}}
\section{}
 % {linear regression}
\include{lecture_notes_05} % {regularized, polynomial, logistic regression}

% [3] Classification
\chapter{Classification}
\rhead{\fancyplain{}{\Title \; 06}}
\section{}
 % {naive bayes, generative vs discriminative}
\include{lecture_notes_07} % {support vector machines}
\include{lecture_notes_08} % {boosting, surrogate losses}
\rhead{\fancyplain{}{\Title \; 09}}
\section{}
 % {decision trees}
\include{lecture_notes_11} % {neural networks and deep learning}

% [4] Nonparametric Methods
\chapter{Nonparametric Methods}
\rhead{\fancyplain{}{\Title \; 12}}
\section{}
 % {k nearest neighbors, kernel density estimation}
\include{lecture_notes_13} % {svm, lin reg (primal + dual), kernels, kernel trick}
\rhead{\fancyplain{}{\Title \; 14}}
\section{}
 % {kernel trick}

% [5] Unsupervised Learning
\chapter{Unsupervised Learning}
\rhead{\fancyplain{}{\Title \; 15}}
\section{}
 % {hierarchical, k means}
\include{lecture_notes_16} % {mixture of gaussians, expectation maximization}
\include{lecture_notes_17} % {latent variable models}
\rhead{\fancyplain{}{\Title \; 18}}
\section{}
 % {graphical models}
\rhead{\fancyplain{}{\Title \; 19}}
\section{}
 % {graphical models}

% [6] Sequence Mdoels
\chapter{Sequence Models}
\include{lecture_notes_20} % {hidden markov models}
\include{lecture_notes_21} % {state space models, time series}

% [7] Representational Learning
\chapter{Representational Learning}
\rhead{\fancyplain{}{\Title \; 22}}
\section{}
 % {feature transformation, random features, pca}
\include{lecture_notes_23} % {pca, ica}

% [8] Reinforcement Learning
%     - markov decision processes
%     - value iteration
%     - Q learning
\chapter{Reinforcement Learning}
\rhead{\fancyplain{}{\Title \; 24}}
\section{}
 % {mdp, value iteration, q learning}
\rhead{\fancyplain{}{\Title \; 25}}
\section{}
 % {q learning for stochastic environment, deep rl}

\end{document}
